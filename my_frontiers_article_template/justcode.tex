% --------------------------------------------------------------------------------------------------
% PREAMBLE
% --------------------------------------------------------------------------------------------------

\documentclass[utf8]{my_class_1_front_vanc}  % using the Vancouver Style (Numbered)

% packages ................................................................
\usepackage{url,hyperref,lineno,microtype,subcaption}
\usepackage[onehalfspacing]{setspace}
\usepackage{comment} % allows using the begin{comment}/end{comment} command

% \linenumbers

% article information .....................................................
\def\Authors{
        Firstname1 Surname1\,$^{1,*}$,
        }
\def\Address{
        $^{1}$ University of Science, New York, USA \\
        $^{2}$ Another Organization, Paris, France
        }

\def\firstAuthorLast{Firstname1 {et~al.}} %use et~al. only if is more than 1 author

% The Corresponding Author should be marked with an asterisk
\def\corrAuthor{Firstname1 Surname1}
\def\corrEmail{address@mail.com}

% Leave a blank line between paragraphs instead of using \\

% sos .....................................................................
\def\keyFont{\fontsize{8}{11}\helveticabold }


%%%%%===========================================================================================%%%%
%%%%%===========================================================================================%%%%
\begin{document}
%%%%%===========================================================================================%%%%

% Page setup ..............................................................
\onecolumn
\firstpage{1}

% Title ...................................................................
\title[This Is The Running Title]{This is The Title of The Article} 

% The fields below will be automatically populated from the information in the preamble
\author[\firstAuthorLast ]{\Authors} 
\address{} 
\correspondence{} 

\extraAuth{}% If there is more than 1 corresponding author, comment this line and uncomment next one
%\extraAuth{corresponding Author2 \\ Laboratory X2, Institute X2, Department X2, Organization X2, Street X2, City X2 , State XX2 (only USA, Canada and Australia), Zip Code2, X2 Country X2, email2@uni2.edu}


\maketitle

% Code section start -------------------------------------------------------------------------------

\section{Code}\label{code}

\verb:\lstinline: prints code snippets inside the line of text.\\
\verb:\lstinputlisting: prints whole files.\\
\verb:\lstlisting: pieces of code which reside in the main file.

\noindent More information here \verb-https://www.overleaf.com/learn/latex/Code_listing-

% The three lines below also work but use different font size, as set in "lstset" in the class file.
% \lstinline:\lstinline: prints code snippets inside the line of text.\\
% \lstinline:\lstinputlisting: prints whole files.\\
% \lstinline:\lstlisting: pieces of code which reside in the main file.\\

% .........................................................................
\subsection{Example 1: Basics of Code Listing}

\begin{verbatim}
This is text enclosed inside \texttt{verbatim} environment.
It is printed directly 
and all \LaTeX{} commands are ignored.
\end{verbatim}

The code below is listed using the {\tt listings} package.

\begin{lstlisting}[language=Python]
def incmatrix(genl1,genl2):
    m = len(genl1)
    n = len(genl2)
    M = None #to become the incidence matrix
    VT = np.zeros((n*m,1), int)  #dummy variable
\end{lstlisting}

The code above was listed using the following instructions:

\begin{verbatim}
\begin{lstlisting}[language=Python]
\end{lstlisting}
\end{verbatim}

% .........................................................................
\subsection{Example 2: Importing Code From a File}

The code below is imported from a file using the line \\
\lstinline:\lstinputlisting[language=Python, firstline=5, lastline=8]{sample_code.py}: \\
As you can see, it only imports the desired lines from that file.

\lstinputlisting[language=Python, firstline=5, lastline=8]{sample_code.py}

% ......................................

The code below is imported from a file using the line \\
\lstinline:\lstinputlisting[language=Python]{sample_code.py}:
% Notice the weird syntax for the thing above, it's bracketed by the colon (:) 
% Neither verb not lstinline use matching pairs of curly braces ({}) to delimit their arguments;
% instead, use any non-letter symbol that doesn't occur in the verbatim material itself
% Do not use an asterisk (*)!

\lstinputlisting[language=Python]{sample_code.py}


% REFERENCES .......................................................................................
% \bibliographystyle{Frontiers-Harvard} 
\bibliographystyle{Frontiers-Vancouver}

% This will create the header "References" which looks just like a section header. 
% But it does not actually create a section (and thus there is bookmark on the PDF)
% \bibliography{test.bib} 

% \nocite{*} 
% This line serves to include ALL the references in the .bib file, even if they have not been cited
% If you have not cited in the body of the article, you can use it to avoid having a blank "section"
% To include only some bibliographical entries, you can use \nocite{key1,key2,...,keyn} to include
% only the entries corresponding to key1,key2,...,keyn.

% If the bib file is working fine, the references should look clean, each one with a number. 

%%%%%===========================================================================================%%%%
\end{document}
%%%%%===========================================================================================%%%%
