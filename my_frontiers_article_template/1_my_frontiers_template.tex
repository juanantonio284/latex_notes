% Based on the template found here:
% https://www.frontiersin.org/design/zip/Frontiers_LaTeX_Templates.zip

% NOTE

% When using latexmk, none of the font changes (e.g. bold and emph) render to PDF. However, it is
% necessary to use latexmk to sort out all the bibliography stuff.

% Once the first version of the pdf (with defective fonts but working bibliography) has come out, 
% pdflatex should be used to get all the font variations). But using pdflatex right away does not
% create a file with a proper references environment.


% --------------------------------------------------------------------------------------------------
% PREAMBLE
% --------------------------------------------------------------------------------------------------

\documentclass[utf8]{my_class_1_front_vanc}  % using the Vancouver Style (Numbered)
% \documentclass[utf8]{FrontiersinHarvard} 
% \documentclass[utf8]{frontiersinFPHY_FAMS} % another Vancouver Style (Numbered) 
% \setcitestyle{square} % for physics and math/stat style
% For some reason, you shouldn't put the .cls extension for the file in the arguments above

% packages ................................................................
\usepackage{url,hyperref,lineno,microtype,subcaption}
\usepackage[onehalfspacing]{setspace}
\usepackage{comment} % allows using the begin{comment}/end{comment} command

\linenumbers

% article information .....................................................
\def\Authors{
        Firstname1 Surname1\,$^{1,*}$,
        Firstname2 Surname2\,$^{1}$, and 
        Firstname3 Surname3\,$^{2}$}
\def\Address{
        $^{1}$ University of Science, New York, USA \\
        $^{2}$ Another Organization, Paris, France
}

\def\firstAuthorLast{Firstname1 {et~al.}} %use et~al. only if is more than 1 author

% The Corresponding Author should be marked with an asterisk
\def\corrAuthor{Firstname1 Surname1}
\def\corrEmail{address@mail.com}

% Leave a blank line between paragraphs instead of using \\

% sos .....................................................................
\def\keyFont{\fontsize{8}{11}\helveticabold }

% Footnotes ----------------------------------------------------------------------------------------
% [This is all still within the preamble]
% For this to work, this line was added to the class file: \usepackage{sepfootnotes}

\sepfootnotecontent{first_note}{This note was made with the {\tt sepfootnotes} package, which helps
 place the footnotes \textbf{sep}arately. This avoids having footnotes in the middle of a paragraph
 in the text file, and makes it easier to read the text file---the actual reading and writing is
 complicated when there are too many \LaTeX\ commands interspersed between the words. (See more on
 this package at https://ctan.org/pkg/sepfootnotes)}


%%%%%===========================================================================================%%%%
%%%%%===========================================================================================%%%%
\begin{document}
%%%%%===========================================================================================%%%%

% Page setup ..............................................................
\onecolumn
\firstpage{1}

% Title ...................................................................
\title[This Is The Running Title]{This is The Title of The Article} 

% The fields below will be automatically populated from the information in the preamble
\author[\firstAuthorLast ]{\Authors} 
\address{} 
\correspondence{} 

\extraAuth{}% If there is more than 1 corresponding author, comment this line and uncomment next one
%\extraAuth{corresponding Author2 \\ Laboratory X2, Institute X2, Department X2, Organization X2, Street X2, City X2 , State XX2 (only USA, Canada and Australia), Zip Code2, X2 Country X2, email2@uni2.edu}


\maketitle


% ABSTRACT environment start -----------------------------------------------------------------------
\begin{abstract}

\textbf{Note} that the header you see above is not a section header, this is not actually a section, it's an environment.

\section{Introduction}\label{abstract_intro}

\textbf{Note} how in the pdf the lines are numbered, even for the section headers, this is because
the abstract's sections are inside the {\tt abstract} environment. The lines of the normal section
headers (those in the rest of the file, outside of the abstract environment) are not numbered.
(It also seems like the normal sections have a different font face and size.)

\section{Objective}\label{abstract_objective}

\textbf{Note} how each paragraph is indented. \textbf{This is in bold letters}. \emph{This is in
italics ( emph).}\\
This sentence---beginning at the T and ending at the period---has 100 characterss, including spaces.

\section{Methods}\label{abstract_methods}

This sentence---beginning at the T and ending at the period---has 100 characterss, including spaces.
This sentence---beginning at the T and ending at the period---has 100 characterss, including spaces.
This sentence---beginning at the T and ending at the period---has 100 characterss, including spaces.

\section{Results}\label{abstract_results}

This sentence---beginning at the T and ending at the period---has 100 characterss, including spaces.
This sentence---beginning at the T and ending at the period---has 100 characterss, including spaces.
This sentence---beginning at the T and ending at the period---has 100 characterss, including spaces.

\section{Conclusion}\label{abstract_conclusion}

This sentence---beginning at the T and ending at the period---has 100 characterss, including spaces.
This sentence---beginning at the T and ending at the period---has 100 characterss, including spaces.
This sentence---beginning at the T and ending at the period---has 100 characterss, including spaces.

\tiny
 \keyFont{ \section{Keywords:} keyword1, keyword2, keyword3 } 
 
This sentence---beginning at the T and ending at the period---has 100 characterss, including spaces.

Note that the line above is indented. It really shouldn't be there, and neither should this text.
It's just meant to show how the {\tt tiny} command still affects things down to this point.

 
\end{abstract} % abstract section end --------------------------------------------------------------


% Introduction section start -----------------------------------------------------------------------

\section{Introduction}

INTRODUCTION. The section header uses all upper-case letters and the ``helveticabold'' command
defined in the class file.

\textbf{Note} how under a {\tt section} command there is no indention in the first paragraph, but
there is indention in the next ones. Also \textbf{Note} how the paragraphs are justified, but the
last line looks smaller (is that {\tt raggedbottom}? or something like that? ).

This sentence---beginning at the T and ending at the period---has 100 characterss, including spaces.
This sentence---beginning at the T and ending at the period---has 100 characterss, including spaces.
This sentence---beginning at the T and ending at the period---has 100 characterss, including spaces.

\subsection{Sub INTRODUCTION Introduction}%\label{}

Sub INTRODUCTION Introduction. The subsection header uses whetever case letters are entered and the
``helveticabold'' command defined in the class file. \textbf{Note} how under a {\tt subsection} or
{\tt subsubsection} command there is indention right from the first paragraph.

\subsubsection{sub sub INTRODUCTION Introduction}%\label{} 

sub sub INTRODUCTION Introduction. The subsubsection header uses whetever case letters are entered
and the ``helvetica'' command defined in the class file. This header \textbf{is not} bold, which
makes sense if you only go down to subsubsection level; but if you go further down to paragraph
level then the paragraph, which below in hierarchy will be bolded and look bigger and look as if
it's above. You can change this behaviour in line 599

\paragraph{This is the header for a paragraph environment} 

The header does not seem to run-in to the text like it usually does but, rather, it behaves like
the {\tt section}, {\tt subsection}, and {\tt subsubsection} headers.

This sentence---beginning at the T and ending at the period---has 100 characterss, including spaces. 
This sentence---beginning at the T and ending at the period---has 100 characterss, including spaces.

\subparagraph{This is the header for a subparagraph environment} 

This sentence---beginning at the T and ending at the period---has 100 characterss, including spaces.
This sentence---beginning at the T and ending at the period---has 100 characterss, including spaces.

This sentence---beginning at the T and ending at the period---has 100 characterss, including spaces.
This sentence---beginning at the T and ending at the period---has 100 characterss, including spaces.
This sentence---beginning at the T and ending at the period---has 100 characterss, including spaces.


% Methods section start ----------------------------------------------------------------------------
\section{Methods}\label{methods}

\href{https://www.frontiersin.org/design/zip/Frontiers_LaTeX_Templates.zip}{This sentence contains a clickable link to download the original latex template.}

This sentence---beginning at the T and ending at the period---has 100 characterss, including spaces.\cite{article}
This sentence---beginning at the T and ending at the period---has 100 characterss, including spaces.\cite{book}
This sentence---beginning at the T and ending at the period---has 100 characterss, including spaces.\\
This sentence contains a footnote before the period\footnote{this is the footnote}.\\
This sentence contains a footnote after the period.\footnote{this is the footnote}\\
Here is another footnote.\sepfootnote{first_note}

% Figure 1 ...................
% \caption{ }

\begin{figure}[htbp]
        \begin{center}
        \includegraphics[width=\linewidth]{homer_giant_donut.jpg}
        \end{center}
    \caption{ This image is 617 x 475 pixels (width x height) }\label{fig:homer_giant_donut}
\end{figure}

% Figure 2 ...................
% \caption{ }

\begin{figure}[htbp]
        \begin{center}
        \includegraphics[width=\linewidth]{mendoza.png}
        \end{center}
    \caption{ Mendozaaaaa! 707 x 527 pixels (width x height) }\label{fig:mendoza}
\end{figure}


% subsection .....................................
% \subsection{}\label{}
% Results section start ----------------------------------------------------------------------------
% \section{Results}\label{results}
% discussion section start -------------------------------------------------------------------------
% \section{Discussion}\label{discussion}
% Conclusion section start -------------------------------------------------------------------------
% \section{Conclusion}\label{conclusion}


\section{Code}\label{code}

\begin{verbatim}
Text enclosed inside \texttt{verbatim} environment 
is printed directly 
and all \LaTeX{} commands are ignored.
\end{verbatim}

\newpage

\begin{lstlisting}[language=TeX]
New text enclosed inside 
    the "lstlisting" environment (from the "listings" package)
        is printed directly 
                and all \LaTeX{} commands are ignored.
\end{lstlisting}



% DECLARATIONS start -------------------------------------------------------------------------------
% the '*' excludes the section from automatic numbering
\section*{Conflict of Interest Statement}

F1L1 straight-up works at WacArnold's---if you want french fries, you gots to go through him.

\section*{Author Contributions}

F1L1 determined the scope, structure, ideas, and conclusions of this study; selected, analysed, and
interpreted the data; and produced various drafts of the manuscript. F2L2 identified the need to
write this article and contributed to research, analysis, and drafting. F3L3 critically reviewed
this article, performed copy-editing duties for the final English manuscript (selected all words
and made all language decisions that affect how the article reads), processed datasets, and
designed tables. All authors read and approved the final manuscript.

\section*{Funding}

This study was executed by Company 1 and partially supported by WacArnold's. 
The funding agreement ensured the authors' independence in designing the study, interpreting the
data, and writing and publishing the present study.

% \section*{Supplemental Data}

%  \href{http://home.frontiersin.org/about/author-guidelines#SupplementaryMaterial}
%   {Supplementary Material} should be uploaded separately on submission, if there are Supplementary
%   Figures, please include the caption in the same file as the figure. LaTeX Supplementary Material
%   templates can be found in the Frontiers LaTeX folder.

\section*{Data Availability Statement}

The original contributions presented in the study are included in the article/supplementary
material, the bibliography contains links to some datasets used, further inquiries can be directed
to the corresponding author.

Below this line you should see a references section (it's not really a section but, rather, some
sort of environment created, at least partially, in the cls file). If you don't see a header that
says "References", try using {\tt latexmk}.


% REFERENCES .......................................................................................
% \bibliographystyle{Frontiers-Harvard} 
\bibliographystyle{Frontiers-Vancouver}

% This will create a the header "References" which looks just like a section header. But it does not
% actually create a section (bookmarked on the PDF)
\bibliography{test.bib} 

% \nocite{*} 
% This line serves to include ALL the references in the .bib file, even if they have not been cited
% If you have not cited in the body of the article, you can use it to avoid having a blank "section"
% To include only some bibliographical entries, you can use \nocite{key1,key2,...,keyn} to include
% only the entries corresponding to key1,key2,...,keyn.

% If the bib file is working fine, the references should look clean, each one with a number. 

%%%%%===========================================================================================%%%%
\end{document}
%%%%%===========================================================================================%%%%
