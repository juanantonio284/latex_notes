%%% Based on Version 3.4 of the template Generated 2022/06/14 %%%

% NOTE: The \emph{} and \textbf{} command do not render to PDF when using this template
%       (also tried \it{}, \textit{}, \em{}).


% For journals "Frontiers in Physics" and "Frontiers in Applied Mathematics and Statistics" 
% \documentclass[utf8]{frontiersinFPHY_FAMS} % Another Vancouver Style (Numbered) 
% \setcitestyle{square} % for physics and math/stat mentioned above

% For some reason, you shouldn't put the .cls extension for the file in the arguments below
% \documentclass[utf8]{FrontiersinHarvard} 
\documentclass[utf8]{test_class} 
% \documentclass[utf8]{FrontiersinVancouver} % for journals using the Vancouver Style (Numbered)

\usepackage{url,hyperref,lineno,microtype,subcaption}
\usepackage[onehalfspacing]{setspace}
\usepackage{comment} % allows using the begin{comment}/end{comment} command

\linenumbers

% Leave a blank line between paragraphs instead of using \\

\def\keyFont{\fontsize{8}{11}\helveticabold }
\def\Authors{
        Firstname1 Lastname1\,$^{1,*}$,
        Firstname2 Lastname2\,$^{1}$, and 
        Firstname3 Lastname3\,$^{2}$}
\def\Address{
$^{1}$ University of Science, New York, USA \\
$^{2}$ Another Organization, Paris, France
}


\def\firstAuthorLast{Firstname1 {et~al.}} %use et al only if is more than 1 author

% The Corresponding Author should be marked with an asterisk
\def\corrAuthor{Firstname1 Lastname1}
\def\corrEmail{address@mail.com}


%%%%%===========================================================================================%%%%
%%%%%===========================================================================================%%%%
\begin{document}
%%%%%===========================================================================================%%%%

\onecolumn
\firstpage{1}

\title[This Is The Running Title]{This is The Title of The Article} 

\author[\firstAuthorLast ]{\Authors} %This field will be automatically populated
\address{} %This field will be automatically populated
\correspondence{} %This field will be automatically populated

\extraAuth{}% If there are more than 1 corresponding author, comment this line and uncomment the next one.
%\extraAuth{corresponding Author2 \\ Laboratory X2, Institute X2, Department X2, Organization X2, Street X2, City X2 , State XX2 (only USA, Canada and Australia), Zip Code2, X2 Country X2, email2@uni2.edu}

\maketitle

% ABSTRACT section start ---------------------------------------------------------------------------
\begin{abstract}

\section{Introduction}\label{abstract_intro}

This line of text has 100 characters. This line of text has 100 characters. This line of text has 1 
\textbf{Notice how, in the pdf, even the section headers are numbered, because they are inside the
 abstract environment}. 
Also notice how each paragraph is indented, but in the normal sections (those in the rest of the
file, outside of the abstract environment) there is no indention.

\section{Objective}\label{abstract_objective}

This line of text has 100 characters. This line of text has 100 characters. This line of text has 1 
This line of text has 100 characters. This line of text has 100 characters. This line of text has 1 
This line of text has 100 characters. This line of text has 100 characters. This line of text has 1 

\section{Methods}\label{abstract_methods}

This line of text has 100 characters. This line of text has 100 characters. This line of text has 1 
This line of text has 100 characters. This line of text has 100 characters. This line of text has 1 
This line of text has 100 characters. This line of text has 100 characters. This line of text has 1 

\section{Results}\label{abstract_results}

This line of text has 100 characters. This line of text has 100 characters. This line of text has 1 
This line of text has 100 characters. This line of text has 100 characters. This line of text has 1 
This line of text has 100 characters. This line of text has 100 characters. This line of text has 1 

\section{Conclusion}\label{abstract_conclusion}

This line of text has 100 characters. This line of text has 100 characters. This line of text has 1 
This line of text has 100 characters. This line of text has 100 characters. This line of text has 1 
This line of text has 100 characters. This line of text has 100 characters. This line of text has 1 

\tiny
 \keyFont{ \section{Keywords:} keyword1, keyword2, keyword3 } 
 
 This line of text has 100 characters. This line of text has 100 characters. This line of text has 1 \\
 The line above shouldn't really be here, it's just meant to show how the {\tt tiny} command still
 affects things down to this point.
 
\end{abstract} % abstract section end --------------------------------------------------------------


% Introduction section start -----------------------------------------------------------------------

\section{Introduction}

INTRODUCTION\\
\textbf{Notice how the paragraphs are justified, but the last line looks smaller (maybe it's not justified).}\\
This line of text has 100 characters. This line of text has 100 characters. This line of text has 1 
This line of text has 100 characters. This line of text has 100 characters. This line of text has 1 
This line of text has 100 characters. This line of text has 100 characters. This line of text has 1 
This line of text has 100 characters. This line of text has 100 characters. This line of text has 1 
This line of text has 100 characters. This line of text has 100 characters. This line of text has 1 

% Methods section start ----------------------------------------------------------------------------
\section{Methods}\label{methods}

This line of text has 100 characters. This line of text has 100 characters. This line of text has 1 
This line of text has 100 characters. This line of text has 100 characters. This line of text has 1 
This line of text has 100 characters. This line of text has 100 characters. This line of text has 1 
This line of text has 100 characters. This line of text has 100 characters. This line of text has 1 
This line of text has 100 characters. This line of text has 100 characters. This line of text has 1 

% Figure 1 ...................
% \caption{ }

\begin{figure}[htbp]
        \begin{center}
        \includegraphics[width=\linewidth]{homer_giant_donut.jpg}
        \end{center}
    \caption{ This image is 617 x 475 pixels (width x height) }\label{fig:homer_giant_donut}
\end{figure}

% Figure 2 ...................
% \caption{ }

\begin{figure}[htbp]
        \begin{center}
        \includegraphics[width=\linewidth]{mendoza.png}
        \end{center}
    \caption{ Mendozaaaaa! 707 x 527 pixels (width x height) }\label{fig:mendoza}
\end{figure}


% subsection .....................................
% \subsection{}\label{}


% Results section start ----------------------------------------------------------------------------
% \section{Results}\label{results}
        

% discussion section start -------------------------------------------------------------------------
% \section{Discussion}\label{discussion}


% Conclusion section start -------------------------------------------------------------------------
% \section{Conclusion}\label{conclusion}


% DECLARATIONS start -------------------------------------------------------------------------------
% the '*' excludes the section from automatic numbering
\section*{Conflict of Interest Statement}

F1L1 straight-up works at WacArnold's---if you want french fries, you gots to go through him.

\section*{Author Contributions}

F1L1 determined the scope, structure, ideas, and conclusions of this study; selected, analysed, and
interpreted the data; and produced various drafts of the manuscript. F2L2 identified the need to
write this article and contributed to research, analysis, and drafting. F3L3 critically reviewed
this article, performed copy-editing duties for the final English manuscript (selected all words
and made all language decisions that affect how the article reads), processed datasets, and
designed tables. All authors read and approved the final manuscript.

\section*{Funding}

This study was executed by Company 1 and partially supported by WacArnold's. 
The funding agreement ensured the authors' independence in designing the study, interpreting the
data, and writing and publishing the present study.

% \section*{Supplemental Data}

%  \href{http://home.frontiersin.org/about/author-guidelines#SupplementaryMaterial}
%   {Supplementary Material} should be uploaded separately on submission, if there are Supplementary
%   Figures, please include the caption in the same file as the figure. LaTeX Supplementary Material
%   templates can be found in the Frontiers LaTeX folder.

\section*{Data Availability Statement}

The original contributions presented in the study are included in the article/supplementary
material, the bibliography contains links to some datasets used, further inquiries can be directed
to the corresponding author.


% REFERENCES .......................................................................................
% \bibliographystyle{Frontiers-Harvard} 
\bibliographystyle{Frontiers-Vancouver}

% This will create a the header "References" which looks just like a section header. But it does not
% actually create a section (bookmarked on the PDF)
\bibliography{test.bib} 


% comment this line when actually citing in the body of the article
\nocite{*} % this includes ALL the references in the .bib file, even if they have not been cited
% it is included in this template just to avoid having a blank "section"
% To include only some bibliographical entries, you can use \nocite{key1,key2,...,keyn} to include
% only the entries corresponding to key1,key2,...,keyn.

% If the bib file is working fine, the references should look clean, each one with a number. 

%%%%%===========================================================================================%%%%
\end{document}
%%%%%===========================================================================================%%%%
