\documentclass{article}

% --------------------------------------------------------------------------------------------------
% PREAMBLE START

\usepackage{comment}
\usepackage{sepfootnotes} % https://ctan.org/pkg/sepfootnotes
\usepackage{color}
\usepackage[usenames,dvipsnames]{xcolor}
\definecolor{gray}{cmyk}{0, 0, 0, 0.25}

\usepackage{listings}
\lstset{
  backgroundcolor=\color{gray!10},    % background color (requires color or xcolor package)
  commentstyle=\color{blue!30!black}, % comment style
  basicstyle=\ttfamily\small,         % font size/family/etc (e.g. basicstyle=\ttfamily\small)
  keywordstyle=\color{red!75!black},  % keyword style  
  stringstyle=\color{green!40!black}, % style of strings in source language
  % frame=single,                       % show a frame surrounding code (see more at note i)
  % numbers=left,                       % Add line numbers on the left side
  % numberstyle=\tiny,                  % style used for line-numbers
  % numbersep=5pt,                      % distance of line-numbers from the code
  } % see more options for this setting in the "listings_package_settings_reference.md" file

% separate footnotes ......................................................

\sepfootnotecontent{gnomon}{A gnomon is the projecting piece on a sundial that shows the time by the
 position of its shadow.}
\sepfootnotecontent{comm_sec_338}{Commands like \texttt{mathit}, \texttt{mathbf}, etc.}
% \sepfootnotecontent{}{}
% \sepfootnotecontent{}{}
% \sepfootnotecontent{}{}


% title .............................................................................................
\title{Chapter 6: Designing It Yourself} % Declares the document's title.
\author{}                                 
\date{}                               
% Note that if you simply omit the `\date` command, the current date will still be printed. 
% To avoid printing a date insert the command with blank arguments, i.e. `\date{ }`.

% PREAMBLE END
\begin{document}
% --------------------------------------------------------------------------------------------------
% --------------------------------------------------------------------------------------------------
\maketitle % Produces the title.

This chapter explains how to specify the visual appearance of a document. Commands specifying the
visual appearance of the document are usually confined to the preamble, either as style
declarations or in the definitions of commands and environments for specifying logical structures.
The notable exceptions are the line- and page-breaking commands of Section 6.2 and the
picture-drawing commands of Section 7.1.

% ..................................................................................................
% ..................................................................................................
\section{Document and Page Style}

% ................................................
\subsection{Document-Class Options}
% pdf 105, page 88

The style of the document can be changed by using the optional argument of the \verb:documentclass:
command to specify style options. Section 2.2.2 describes the following options: 
\verb:11pt, 12pt, twoside, and twocolumn:

Three notable \emph{standard} options are:

\begin{itemize}
   
   \item {\tt titlepage} causes the \verb:maketitle: command to generate a separate title page.
    (This option is the default for the {\tt report} document class.)
   
   \item An abstract is made with the \texttt{abstract} environment. It is placed on a separate page
    if the \texttt{titlepage} document-class option is in effect; otherwise, it acts like an
    ordinary displayed-paragraph environment.

   \item {\tt leqno} causes the formula numbers produced by the {\tt equation} and {\tt eqnarray}
    environments to appear on the left instead of the right
   
   \item {\tt fleqn} causes formulas to be aligned on the left, a fixed distance from the left
    margin, instead of being centered
\end{itemize}

A complete list of the standard options appears in Section C.5.1 % pdf 194, page 176

The {\tt twocolumn} option is used for a document that consists mostly of double-column pages. You 
may want double-column pages for just part of the document, such as a glossary. In that case, 
the \verb:twocolumn: declaration starts a new page and begins producing two-column output. 
The inverse \verb:onecolumn: declaration starts a new page and begins producing single-column 
output.

{\tt twocolumn} sets various style parameters, such as the amount of paragraph indentation, to
values appropriate for such pages. It does not change any style parameters except the ones required
to produce two-column output.

% ................................................
\subsection{Page Styles}
% pdf 106, page 89

A page of output consists of three units: the head, the body, and the foot. The body consists of
everything between the head and foot: the main text, footnotes, figures, and tables.

In most pages of this book, the head contains a page number, a chapter or section title, and a
horizontal line, while the foot is empty; but in the table of contents and the preface, the page
head is empty and the foot contains the page number.

Page headings may contain additional information to help the reader. They are most useful in
two-sided printing, since headings on the two facing pages convey more information than the single
heading visible with one-sided printing. Page headings are generally not useful in a short
document, where they tend to be distracting rather than helpful. 

\subsubsection{Page Numbers}

You can specify arabic page numbers with a \verb:pagenumbering(arabic): 
command and roman numerals with a \verb:pagenumbering(roman): command, the default being arabic 
numbers.

The \verb:pagenumbering: declaration resets the page counter (the current page is numbered 1). 
To begin a document with pages i, ii, etc. and have the first chapter start with page 1, 
put \verb:pagenumbering(roman): anywhere before the beginning of the text and 
\verb:pagenumbering(arabic): right after the first \verb:chapter: command.

We can make the pages start with any number we want by the command \verb:\setcounter{page}{number}:
where number is the page number we wish the current page to have.

\subsubsection{Page Styles}

The \emph{page style} determines what goes into the head and foot; it is specified with
a \verb:\pagestyle: declaration having the style's name as its argument.

The \verb:\pagestyle: declaration obeys the normal scoping rules. What goes into a page's head and
foot is determined by the page style in effect at the end of the page, so the \verb:\pagestyle:
command usually comes after a command like \verb:\chapter: that begins a new page.

The \verb:\thispagestyle: command is like \verb:\pagestyle: except it affects only the current page.
Some commands, such as \verb:\chapter:, change the style of the current page. These changes can be
overridden with a subsequent \verb:\thispagestyle: command. 

There are four standard page styles:

\begin{itemize}
   \item \texttt{plain}: the page number is in the foot and the head is empty. It is the default
    page style for the article and report document classes

   \item \texttt{empty}: the head and foot are both empty (\LaTeX\ still assigns each page a number,
    but the number is not printed)
   
   \item \texttt{headings}: the page number and other information, determined by the document style,
    is put in the head; the foot is empty
   
   \item \texttt{myheadings}: similar to the headings page style, except you specify the ``other
    information'' that goes in the head, using the \verb:\markboth: and \verb:\markright: commands
    described below
\end{itemize}

\subsubsection{Configuring headings and myheadings}

The contents of the page headings in the \verb:headings: and \verb:myheadings: styles are set by the 
commands \\
\verb:\markboth{left_head}{right_head}: and \verb:\markright{right_head}:

The left\_head and right\_head arguments specify the information to go in the page
heads of left-hand and right-hand pages, respectively. 

In two-sided printing, specified with the \texttt{twoside} document-class option, the even-numbered
pages are the left-hand ones and the odd-numbered pages are the right-hand ones.

In one-sided printing, all pages are considered to be right-hand ones. In the headings page style,
the sectioning commands choose the headings for you; Section C.5.3 explains how to use 
\verb:markboth: and \verb:markright: to override their choices.

In the \verb:myheadings: option, you must use these commands to set the headings yourself. The
arguments of \verb:markboth: and \verb:markright: are processed in LR mode; they are moving
arguments.

% ................................................
\subsection{The Title Page and Abstract}
% pdf 107, page 90

The \verb:\maketitle: command produces a title page when the \texttt{titlepage} document-class
option is in effect.\footnote{Described in Sections 2.2.2 and C.5.4.}

You can also create your own title page with the \texttt{titlepage} environment. This environment
creates a page with the empty page style, so it has no printed page number or heading. It causes
the following page to be numbered 1. You are completely responsible for what appears on a title
page made with the \texttt{titlepage} environment. 

The following commands and environments are useful in formatting a title page: 

\begin{itemize}
  \item the type-size-changing commands of Section 6.7.1
  \item the type-style-changing commands of Section 3.1
  \item the center environment of Section 6.5
  \item the \verb:\today: command, described in Section 2.2.1, to generate today's date
\end{itemize}

An abstract is made with the \texttt{abstract} environment. It is placed on a separate page if
the \texttt{titlepage} document-class option is in effect; otherwise, it acts like an ordinary
displayed-paragraph environment.

% Omitted section 6.1.4 \section{Customizing the Style}

% ..................................................................................................
% ..................................................................................................
\section{Line and Page Breaking}

\LaTeX\ usually does a good job of breaking text into lines and pages, but it sometimes needs help.
Don't worry about line and page breaks until you prepare the final version. Most of the bad breaks
that appear in early drafts will disappear users waste time formatting when they as you change the
text. Many should be writing. 
\textbf{Don't worry about line and page breaks until you prepare the absolutely final version}.

% ................................................
\subsection{Line Breaking}\label{sub-line-break}
% pdf 110 page 93

You can teach \TeX\ how to hyphenate words by putting one or more \verb:hyphenation: commands in
the preamble. For example, the command \\
\verb:hyphenation\{gno-mon gno-mons gno-mon-ly}: tells \TeX\ how to hyphenate 
gnomon\sepfootnote{gnomon}, gnomons, and gnomonly---but it still won't know how to hyphenate 
gnomonic.

Not all line-breaking problems can be solved by hyphenation. Sometimes there is just no good way to
break a paragraph into lines. \TeX\ is normally very fussy about line breaking---it lets you solve
the problem rather than producing a paragraph that doesn't meet its high standards. There are three
things you can do when this happens.

\begin{itemize}
   \item Rewrite the paragraph---if willing and able
   \item Use a {\tt sloppypar} environment or \verb:sloppy: declaration, which direct \TeX\ not
    to be so fussy about where it breaks lines
   \item Use a \verb:linebreak: command, which forces \TeX\ to break the line at that spot
\end{itemize}

% .......................
\subsubsection{linebreak and nolinebreak}

The \verb:linebreak: is usually inserted right before the word that doesn't fit. An optional
argument converts the \verb:linebreak: command from a demand to a request. The argument must be a
digit from 0 through 4, a higher number denoting a stronger request. 

\begin{itemize}
   
   \item \verb:linebreak[O]: allows \TeX\ to break a line where it would normally be forbidden but
    does not necessarily encourage or discourage it
   
   \item \verb:linebreak[4]: command means the same as an ordinary \verb:linebreak: command and
    forces the line break
        
\end{itemize}
 
Although unwanted line breaks are usually prevented with the \~ and \verb:mbox: commands described
in Section 2.2.1, \TeX\ also provides a \verb:nolinebreak: command that forbids from breaking the
line at that point. An optional argument converts the \verb:nolinebreak: command from a demand to a
request. The argument must be a digit from 0 through 4, a higher number denoting a stronger
request.

\begin{itemize}
   
   \item \verb:nolinebreak[4]: is equivalent to \verb:nolinebreak: and \textbf{forbids} a line break
    at that point
   
   \item \verb:nolinebreak[O]: is equivalent to \verb:linebreak[O]:
\end{itemize}

A \verb:linebreak: command causes \TeX\ to justify the line, stretching the space between words so
that the line extends to the right margin. The \verb:newline: command ends a line without
justifying it.

% ................................................
\subsection{Page Breaking}
% pdf 113 page 96

As with line breaking, sometimes \TeX\ can find no good place to start a new page. A bad page break
usually causes \TeX\ to put too little rather than too much text onto the page.

As with line breaking, \TeX\ provides commands to demand or prohibit a page break, with an optional
argument transforming the commands to suggestions. (The \TeX\ page-breaking commands are analogous
to the line-breaking commands described in \ref{sub-line-break} above.)

A \verb:pagebreak: command \emph{within a paragraph} insists that \TeX\ start a new page after
the line in which the command appears, and \verb:nopagebreak: suggests rather strongly
that \TeX\ not start a new page there. (When used between paragraphs, the commands apply to that
point.)

You will sometimes want to squeeze a little more text on a page than \TeX\ thinks you should. The
best way of doing this is to make room on the page by removing some vertical space, using the
commands of Section 6.4.2.

If that doesn't work, you can try using a \verb:nopagebreak: command to prevent \TeX\ from
breaking the page before you want it to. However, \TeX\ often becomes adamant about breaking a page
at a certain point, and it will not be deterred by \verb:nopagebreak:. When this happens, you
can put extra text on the page as follows:

\begin{itemize}
   \item Add the command \verb:\enlargethispage*{1000pt}: to the text on the current page, before
    the point where \TeX\ wants to start a new page
   
   \item Add a \verb:pagebreak: command to break the page where you want
\end{itemize}

Squeezing in extra text in this way will make the page longer than normal, which may look bad in
two-sided printing. Section C.12.2 describes a less heavy-handed approach. But don't worry about
page breaks until you prepare the ultimate, absolutely final, no-more-changes (really!) version.

The \verb:newpage: command is the analog of \verb:newline:, creating a page that ends prematurely
right at that point. Even when a \verb:flushbottom: declaration is in effect, a shortened page is
produced.

The \verb:clearpage: command is similar to \verb:newpage:, except that any leftover figures or
tables are put on one or more separate pages with no text. The \verb:chapter: and \verb:include:
commands (Section 4.4) use \verb:clearpage: to begin a new page.

Adding an extra \verb:newpage: or \verb:clearpage: command will not produce a blank page; two such
commands in a row are equivalent to a single one. To generate a blank page, you must put some
invisible text on it, such as an empty \verb:mbox:.

The page-breaking commands can be used only where it is possible to start a new page---that is, in
paragraph mode and not inside a box (Section 6.4.3). They are all fragile.

\paragraph{Extra:} When using the {\tt twoside} style option for two-sided printing, you may want to
 start a sectional unit on a right-hand page. 
 The \verb:cleardoublepage: command is the same as \verb:clearpage: except that it produces a blank 
 page if necessary so that the next page will be a right-hand (odd-numbered) one. 
 When used in two-column format, the \verb:newpage: and \verb:pagebreak: commands start a new column 
 rather than a new page. 
 However, the \verb:clearpage: and \verb:cleardoublepage: commands start a new page.


% ..................................................................................................
% ..................................................................................................
\section{Numbering}
% pdf 114 page 97

Every number that \LaTeX\ generates has a counter associated with it. The name of the counter is
usually the same as the name of the environment or command that produces the number, except with
no \textbackslash. Below is a list of the counters used to control numbering by \LaTeX's standard
document styles.

\begin{verbatim}
part paragraph figure enumi
chapter subparagraph table enumii
section page footnote enumiii
subsection equation mpfootnote enumiv
subsubsection
\end{verbatim}

The counters {\tt enumi ... enumiv} control different levels of {\tt enumerate} environments,
{\tt enumi} for the outermost level, {\tt enumii} for the next level, etc.

The {\tt mpfootnote} counter numbers footnotes inside a minipage environment (Section 6.4.3). 

In addition to these, an environment created with the \verb:newtheorem: command (Section 3.4.3) has
a counter of the same name unless an optional argument specifies that it is to be numbered the same
as another environment. There are also some other counters used for document-style parameters; they
are described in Appendix C.

The \verb:setcounter: command sets the value of a counter, and \verb:addtocounter: increments it by
a specified amount. 

More on counters, including the \verb:newcounter: command, on page 98.


% ..................................................................................................
\section{Length, Spaces, and Boxes}

In visual design, one specifies how much vertical space to leave above a chapter heading, how wide a
line of text should be, and so on. This section describes the basic tools for making these
specifications.

% ................................................
\subsection{Length}
% pdf 116 page 99

A line width is specified by giving a \emph{length} as an argument to the appropriate formatting
command. A length can be specified in fixed units like inches, centimeters, milimeters, or points
(used by printers). It can also be specified in units of length that depend upon the appropriate
style parameters. The simplest such units are the {\tt em} and the {\tt ex}, which depend upon the
font (the size and style of type). A {\tt 1em} length is about equal to the width of the letter
``M'', and {\tt 1ex} is about the height of the letter ``x''. The em is best used for horizontal
lengths and the ex for vertical lengths.

\subsubsection{Length Commands}

In addition to writing explicit lengths such as {\tt 1in} or {\tt 3.5em}, you can also express
lengths with length commands. A length command has a value that is a length. 

For example: \verb:parindent: is a length command whose value specifies the width of the indentation
at the beginning of a normal paragraph. Typing \verb:parindent: as the argument of a command is
equivalent to typing the current value of \verb:parindent:. Typing \verb:2.5\parindent: will give a
length that is 2.5 times as large; \verb:-2.5\parindent: is the negative of that length.

A length such as 1.5em or \verb:parindent: is a \textbf{rigid} length. Specifying a space of width
1cm always produces a one-centimeter-wide space. (It may not be exactly one centimeter wide because
your output device might unifordly change all dimensions-for example, enlarging them by 5\%.) 

However, there are also \textbf{rubber} lengths that can vary. Space specified with a rubber length
can stretch or shrink as required. For example, \TeX\ justifies lines (produces an even right
margin) by stretching or shrinking the space between words to make each line the same length. A
rubber length has a natural length and a degree of elasticity. Of particular interest is the
special length command \verb:fill: that has a natural length of zero but is infinitely
stretchable---and tends to expand as far as it can. The use of such stretchable space is described
in the next section, \ref{sec-spaces}.

Most lengths used in \LaTeX\ are rigid. Unless a length is explicitly said to be rubber, you can
assume it is rigid. All length commands are robust; a \verb:protect: command should never precede a
length command.

(The rest of this subsection covers some of \LaTeX length parameters---i.e. length commands that
define a document's style parameters; others are given in Appendix C. By expressing lengths in
terms of these parameters, you can define formatting commands that work properly with different
styles.)
% Here (pdf 118 page 100) there are is a list of commands and a short explanation of how they work:
% parindent, textwidth, textheight, parskip, baselineskip, 

\LaTeX\ provides the following declarations for changing the values of length commands and for
 creating new ones. These declarations obey the usual scoping rules.

\begin{itemize}
   \item \verb-\newlength-: defines a new length command
   \item \verb-\setlength-: sets the value of a length command
   \item \verb-\addtolength-:increments the value of a length command by a specified amount
   \item \verb-\settowidth-:sets the value of a length command equal to the width of a specified
    piece of text
   \item \verb:\settoheight:, \verb-\settodepth-: these commands act like \verb:\settowidth:, except
    they set the value of the length command to the height and depth, respectively, of the text
\end{itemize}
 
Height and depth are explained in the Boxes section (\ref{sec-boxes}).


% ..................................................................................................
% ..................................................................................................
\section{Spaces}\label{sec-spaces}
% pdf 118 page 101

A horizontal space is produced with the \verb:hspace: command. Think of \verb:hspace: as
making a blank ``word'', with spaces before or after it producing an interword space.

Here\hspace{.5in}is a 0.5 inch space.

Here  \hspace{.5in}is a 0.5 inch space.

Here  \hspace{.5in}  is a 0.5 inch space.

Negative space is a backspace---like this.\hspace{.5in}///// 
% this command is not working as in textbook

\bigskip

\noindent The text above is created with the following code: 

\begin{lstlisting}[language={[LaTeX]TeX}]
Here\hspace{.5in}is a 0.5 inch space.

Here  \hspace{.5in}is a 0.5 inch space.

Here  \hspace{.5in}  is a 0.5 inch space.

Negative space is a backspace---like this.\hspace{.5in}/////
\end{lstlisting}

\TeX\ removes space from the beginning or end of each line of output text, except at the beginning
and end of a paragraph---including space added with \verb:hspace:. 
The \verb:hspace*: command is the same as \verb:hspace: except that the space it produces is
never removed, even when it comes at the beginning or end of a line. 
The \verb:hspace: and \verb:hspace*: commands are robust.

The \verb:vspace: command produces vertical space. It is most commonly used between paragraphs;
when used within a paragraph, the vertical space is added after the line in which the 
\verb:vspace: appears.

Just as it removes horizontal space from the beginning and end of a line, \TeX\ removes vertical
space that comes at the beginning or end of a page. The \verb:vspace*: command creates vertical
space that is never removed


% ..................................................................................................
\section{Boxes}\label{sec-boxes}
% pdf 120 page 103

A \emph{box} is a chunk of text that \LaTeX\ treats as a unit, just as if it were a single letter.
No matter how big it is, \LaTeX\ will never split a box across lines or across pages.
The \verb:\mbox: command introduced in Section 2.2.1 prevents its argument from being split across
lines by putting it in a box.

Many other \LaTeX\ commands and environments produce boxes. For example, the \texttt{array} and 
\texttt{tabular} environments (Section 3.6.2) both produce a single box that can be quite big, as
does the picture environment described in Section 7.1.

A box has a \textbf{reference point}, which is at its left-hand edge. \TeX\ produces lines by
putting boxes next to one another with their reference points aligned, as shown in Figure 6.1. The
figure also shows how the width, height, and depth of a box are computed.

\LaTeX\ provides additional commands and environments for making three kinds of boxes: 
\emph{LR boxes}, in which the contents of the box are processed in LR mode; 
\emph{parboxes}, in which the contents of the box are processed in paragraph mode; and 
\emph{rule boxes}, consisting of a rectangular blob of ink.

A box-making command or environment can be used in any mode; \LaTeX\ will use the declarations in
effect at that point when typesetting the box's contents. For example, the contents of a box
appearing in the scope of an \verb:\em: declaration will be emphasized. However, box-making
commands that appear in a mathematical formula are not affected by the commands described in
Section 3.3.8 that change the type style in the formula\sepfootnote{comm_sec_338}. (Since the input
that produces a box's contents is either the argument of a box-making command or the text of a
box-making environment, any declarations made inside it are local to the box.)
% small example on pdf 120 page 103

A box is often displayed on a line by itself. This can be done by treating the box as a formula and
using the \texttt{displaymath} environment \verb:(\[ ... \]):. The center environment described in
Section 6.5 can also be used.
% starting at page 104, there is a dedicated subsection for each of LR boxes, parboxes, and rule
% boxes (note sure how useful it is ...)

% ................................................
\subsection{Raising and Lowering Boxes}
% pdf 124 page 107

The \verb:\raisebox: command raises text by a specified length (a negative length lowers the text).
It makes an LR box, just like the \verb:\mbox: command. 

% example in page

It is sometimes useful to change how big \TeX\ thinks a piece of text is without changing the text.
By changing the apparent height of text, you change how much space \TeX\ leaves for it. This is
sometimes used to eliminate space above or below a formula or part of a formula.

% The \verb:\makebox: command tells \TeX\ how wide the text is, while a strut can increase the text's apparent height but cannot decrease it.

Optional arguments of \verb:\raisebox: tell \LaTeX\ how tall it should pretend that the text is.

For example, the command \verb:\raisebox{.4ex}[1.5ex][.75ex]{\em text}: raises the text by 
\texttt{.4ex}, and it also makes \TeX\ think that it extends \texttt{1.5ex} above the bottom of the
line, and \texttt{.75ex} below the bottom of the line. (The bottom of the line is where most
characters sit; a letter like y extends below it.) If you omit the second optional argument, \TeX\
will think the text extends as far below the line as it actually does.

% ................................................
% Omitted \subsection{Saving Boxes}

\begin{comment}

% The example below is so hard to follow that it doesn't seem worth the trouble, commented all

% ................................................
\section{Formatting with Boxes}
% pdf 125 page 108

I will illustrate the power of box commands with a silly example: defining a \verb:\face: command to put a funny face around a word.

The face is composed of two diamonds (\diamond) produced with the \verb:\diamond: command to create the eyes, and a smile (\smile) produced with \verb:\smile:. The trick, of course, is positioning these symbols correctly relative to the word being ``faced''.

The left eye of the face (the one near the ``f'') is positioned above the word and 15\% of the way from the word's left edge to its right edge. 

First, we see how to put a piece of text a fixed distance to the right of the left-hand edge of a
word. Recall that if the specified width of a \verb:\makebox: is less than the width of its
contents, the contents extend outside the box; so, in the extreme case of a zero-width box, all the
box's contents lie to the right (and on run on top of other text)

\textbf{Example}
% pdf 126 page 109

This code in the box below is supposed to render but it creates an error. 
% The gnuuuuu began to \makebox[Opt][1]{\hspace{1cm}X} moooooooooooo.

\begin{lstlisting}[language={[LaTeX]TeX}]
The gnuuuuu began to \makebox[Opt][1]{\hspace{1cm}X}moooooooooooo.
\end{lstlisting}

Because the box produced by this \makebox command has zero width, it doesn't change the position of
the ``moo...''

% example ends here, back to face .............

We want the face's left eye to be shifted to the right by 0.15 times the width of the word and
raised by the height of the word. So the \verb:\face: command must measure the width and height of
its argument using the \verb:\settoheight: and \verb:\settowidth: commands. These lengths are saved
in two new length commands, \verb:\faceht: and \verb:\facewd:, that are defined with 
\verb:\newlength:.

The eye is raised with the \verb:\raisebox: command. 

Before figuring out how to draw the rest of the face, we define a \verb:\lefteye: command
so \verb:\lefteye{\emph{moooo}}: produces moooo. (With a diamond above the ``m''.)

\end{comment}


% ..................................................................................................
% \section{}\label{}
% pdf  page

% ................................................
% \subsection{}
% pdf  page


% ................................................
% \subsection{}
% pdf  page

% ................................................
% \subsection{}
% pdf  page


% ................................................
% \subsection{}
% pdf  page


% --------------------------------------------------------------------------------------------------
% --------------------------------------------------------------------------------------------------
\end{document}               % End of document.
% --------------------------------------------------------------------------------------------------
% --------------------------------------------------------------------------------------------------
