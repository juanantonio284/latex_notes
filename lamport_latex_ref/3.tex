\documentclass{article}

% --------------------------------------------------------------------------------------------------
% PREAMBLE START

% separate footnotes ...............................................................................
\usepackage{sepfootnotes} % https://ctan.org/pkg/sepfootnotes

\sepfootnotecontent{modes_tex}{
Paragraph mode corresponds to the vertical and ordinary horizontal modes in 
\emph{The \TeX book}, and LR mode is called restricted horizontal mode there. 
\LaTeX\ also has a restricted form of LR mode called picture mode that is described in Section 7.1.
}



% title ............................................................................................
\title{Chapter 3} % Declares the document's title.
\author{}                                 
\date{}                               

% Note that if you simply omit the `\date` command, the current date will still be printed. To avoid printing a date insert the command `\date{ }`, i.e. the command with blank arguments.

% --------------------------------------------------------------------------------------------------
% PREAMBLE END


\begin{document}
% --------------------------------------------------------------------------------------------------
% --------------------------------------------------------------------------------------------------

\maketitle % Produces the title.

[This chapter is titled ``Carrying On'' which doesn't really describe what's in the chapter, and the
information below doesn't belong to any section, it's just presented with no preamble at the
beginning of the chapter. But I found it interesting ...] 

% ..................................................................................................
\section{Modes}
% pdf 52, page 36

As \TeX\ processes your input text, it is always in one of three modes: paragraph mode, math mode,
or left-to-right mode (called LR mode for short)\sepfootnote{modes_tex}. 

\begin{itemize}
   
   \item Paragraph mode is \TeX's normal mode---the one it's in when processing ordinary text. In
    paragraph mode, \TeX\ regards your input as a sequence of words and sentences to be broken into
    lines, paragraphs, and pages.

    \item \TeX\ is in math mode when it's generating a mathematical formula. More precisely: it
     enters math mode upon encountering a command like 
     \$ or 
     \verb:\(: or
     \verb:\[: or
     \verb:\begin{equation}: 
     that begins a mathematical formula; and it leaves math mode after finding the corresponding
     command that ends the formula. When \TeX\ is in math mode it considers letters in the input
     file to be mathematical symbols and ignores any space characters 
     between them—e.g. ``\emph{is}`` would be treated as the product of \emph{i} and \emph{s}.
     
    \item In left-to-right mode, as in paragraph mode, \TeX\ considers your input to be a string of
     words with spaces between them. However, unlike paragraph mode, \TeX\ produces output that
     keeps going from left to right; \textbf{it never starts a new line in LR mode}.
     The \verb:\mbox: command (Section 2.2.1) causes \TeX\ to process its argument in LR mode,
     which is what prevents the argument from being broken across lines.

\end{itemize}

Different modes can be nested within one another as seen, for example, when you put an 
\verb:\mbox: command inside a mathematical formula.

\subsection*{Example}\label{example}

Consider this expression: \( y > z \mbox{ if $x^{2}$ real} \).

\noindent Made with this code \verb:\( y > z \mbox{ if $x^{2}$ real} \).: \\

\noindent When processing this expression, \LaTeX\ enters and exits {\tt math mode} and
 {\tt left-to-right mode}. (For the expressions below, the blank spaces between the letters are
 also processed in each mode.) 
 
\begin{itemize}
   
   \item when processing \verb:y > z: (i.e. \textvisiblespace y \textvisiblespace > \textvisiblespace z \textvisiblespace) \TeX\ is in math mode
   
   \item when processing \verb: if : and \verb: real: \TeX\ is in LR mode
   
   \item when processing \verb:x^{2}: \TeX\ is in math mode
   
   \item The space between ``z'' and ``if'' is produced by the first space in the argument for
    {\tt mbox}

   \item The space in \verb:real} \): is processed in math mode, so it produces no space between
    ``real'' and ``.''
    
\end{itemize}


% ..................................................................................................
\section{Changing the Type Style}
% pdf 52, page 36

Type style is used to indicate logical structure. In this book, emphasized text appears in 
\textit{italic} style type and \LaTeX\ input in \texttt{typewriter} style.

None of the text-producing commands or declarations can be used in math mode. (Section 3.3.8
explains how to change type style in a mathematical formula.)

Type style is a visual property. Commands to specify visual properties belong not in the text, but
in the definitions of commands that describe logical structure. Section 3.4 explains how to define
your own commands for the logical structures in your document. 

% For example, suppose you want the names of genera to appear in italic in your book on African
% mammals. Don't use \verb:\textit: throughout the text; instead, define a \verb:\genus: command and
% write something like \linebreak \verb:\genus{Connochaetes} seems to pop up ...: Then, if you decide
% that Connochaetes and all other genera should appear in slanted rather than italic type, you just
% have to change the definition of \verb:\genus:.

For example, suppose you want the names of genera to appear in italic in your book on African
mammals. Don't use \verb:\textit: throughout the text; instead, define a \verb:\genus: command and
write something like \\
\verb:\genus{Connochaetes} seems to pop up ...: \\
\noindent Then, if you decide that Connochaetes and all other genera should appear in slanted rather
 than italic type, you just have to change the definition of \verb:\genus:.


% --------------------------------------------------------------------------------------------------
% --------------------------------------------------------------------------------------------------
\end{document}               % End of document.
% --------------------------------------------------------------------------------------------------
% --------------------------------------------------------------------------------------------------
