\documentclass{article}

% --------------------------------------------------------------------------------------------------
% PREAMBLE START

% separate footnotes ...............................................................................
\usepackage{sepfootnotes} % https://ctan.org/pkg/sepfootnotes

\sepfootnotecontent{modes_tex}{
Paragraph mode corresponds to the vertical and ordinary horizontal modes in 
\emph{The \TeX book}, and LR mode is called restricted horizontal mode there. 
\LaTeX\ also has a restricted form of LR mode called picture mode that is described in Section 7.1.
}



% title ............................................................................................
\title{Chapter 3} % Declares the document's title.
\author{}                                 
\date{}                               

% Note that if you simply omit the `\date` command, the current date will still be printed. To avoid printing a date insert the command `\date{ }`, i.e. the command with blank arguments.

% --------------------------------------------------------------------------------------------------
% PREAMBLE END


\begin{document}
% --------------------------------------------------------------------------------------------------
% --------------------------------------------------------------------------------------------------

\maketitle % Produces the title.

[This chapter is titled ``Carrying On'' which doesn't really describe what's in the chapter, and the
information below doesn't belong to any section, it's just presented with no preamble at the
beginning of the chapter. But I found it interesting ...] 

% ..................................................................................................
\section{Modes}

As \TeX\ processes your input text, it is always in one of three modes: paragraph mode, math mode,
or left-to-right mode (called LR mode for short)\sepfootnote{modes_tex}. 

\begin{itemize}
   
   \item Paragraph mode is \TeX's normal mode-the one it's in when processing ordinary text. In
    paragraph mode, \TeX\ regards your input as a sequence of words and sentences to be broken into
    lines, paragraphs, and pages.

    \item \TeX\ is in math mode when it's generating a mathematical formula. More precisely, it
     enters math mode upon encountering a command like 
     \$ or 
     {\tt \textbackslash (} or
     {\tt \textbackslash [} or 
     {\tt \textbackslash begin\{equation\} } 
     that begins a mathematical formula, and it leaves math mode after finding the corresponding
     command that ends the formula. When \TeX\ is in math mode it considers letters in the input
     file to be mathematical symbols and ignores any space characters 
     between them—e.g. \emph{is} would be treated as the product of \emph{i} and \emph{s}.
     
    \item In LR mode, as in paragraph mode, \TeX\ considers your input to be a string of words with
     spaces between them. However, unlike paragraph mode, \TeX\ produces output that keeps going
     from left to right; it never starts a new line in LR mode. The {\tt \textbackslash mbox} command
     (Section 2.2.1) causes \TeX\ to process its argument in LR mode, which is what prevents the
     argument from being broken across lines.

\end{itemize}

Different modes can be nested within one another as seen, for example, when you put an 
{\tt \textbackslash mbox}~command inside a mathematical formula.

% --------------------------------------------------------------------------------------------------
% --------------------------------------------------------------------------------------------------
\end{document}               % End of document.
% --------------------------------------------------------------------------------------------------
% --------------------------------------------------------------------------------------------------
