\documentclass{article}

% --------------------------------------------------------------------------------------------------
% PREAMBLE START

\usepackage{sepfootnotes} % https://ctan.org/pkg/sepfootnotes
\usepackage{color}
\usepackage[usenames,dvipsnames]{xcolor}
\definecolor{gray}{cmyk}{0, 0, 0, 0.25}

\usepackage{listings}
\lstset{
  backgroundcolor=\color{gray!10},    % background color (requires color or xcolor package)
  commentstyle=\color{blue!30!black}, % comment style
  basicstyle=\ttfamily\small,         % font size/family/etc (e.g. basicstyle=\ttfamily\small)
  keywordstyle=\color{red!75!black},  % keyword style  
  stringstyle=\color{green!40!black}, % style of strings in source language
  % frame=single,                       % show a frame surrounding code (see more at note i)
  % numbers=left,                       % Add line numbers on the left side
  % numberstyle=\tiny,                  % style used for line-numbers
  % numbersep=5pt,                      % distance of line-numbers from the code
  } % see more options for this setting in the "listings_package_settings_reference.md" file
  
  
% separate footnotes ......................................................

\sepfootnotecontent{error}{
An error in your input file could produce an error in one of the special cross-referencing files.
The error in the cross-referencing file will not manifest itself until that file is read, the next
time you run \LaTeX. Section 8.1 explains how to recognize such an error.
}

\sepfootnotecontent{firstrun}{the next time \LaTeX\ is run on the same input file}

\sepfootnotecontent{produce-list-keys}{https://tex.stackexchange.com/questions/525394/get-a-list-of-all-labels-in-a-tex-document}

\sepfootnotecontent{fragile}{https://tex.stackexchange.com/questions/4736/what-is-the-difference-between-fragile-and-robust-commands-when-and-why-do-we-n}


% title .............................................................................................
\title{Chapter 4: Moving Information Around} % Declares the document's title.
\author{}                                 
\date{}                               
% Note that if you simply omit the `\date` command, the current date will still be printed. To avoid printing a date insert the command `\date{ }`, i.e. the command with blank arguments.

% PREAMBLE END
\begin{document}
% --------------------------------------------------------------------------------------------------
% --------------------------------------------------------------------------------------------------
\maketitle % Produces the title.

\LaTeX\ often has to move information from one place to another. For example, the information
contained in a table of contents comes from the sectioning commands that are scattered throughout
the input file. Similarly, the command that generates a cross-reference to an equation must get the
equation number from an {\tt equation} environment which may occur several sections later. 

In other words, references appear in different places than the commands which create the information
for those references. So \LaTeX\ requires two passes over the input to move information around: one
pass to find the information and a second pass to put it into the text—it occasionally even
requires a third pass. To compile a table of contents, for example, one pass determines the titles
and starting pages of all the sections, and a second pass puts this information into the table of
contents. But instead of making two passes every time it is run, \LaTeX\ reads your input file only
once and saves the cross-referencing information in special files for use the next time. 

For example: if {\tt sample2e.tex} had a command to produce a table of contents \LaTeX\ would write
the necessary information into the file {\tt sample2e.toc}; then it would use the information in
that file to produce the table of contents in the typeset document. After that, the
{\tt sample2e.toc} file would be re-written to a newer version (this version will be used to
produce the table of contents the next time \LaTeX\ is run) on the tex file. But notice that this
``newer'' version will actually be a version from a previous execution, meaning that \LaTeX's
cross-referencing information is always ``old''. This will be noticeable mainly when you are first
writing the document—for example, a newly added section won't be listed in the table of
contents—but running it again on the same input will correct any errors\sepfootnote{error}.


% ..................................................................................................
\section{Cross-References}

One reason for numbering things like figures and equations is to refer the reader to them, as in:
``See Figure 3 for more details''. You don't want the character `3' to actually be written in the
input file because adding another figure might make the figure become ``Figure 4'', and require
changing the `3' to a `4' everywhere it appears. Instead, you can assign a \emph{key} of your
choice to the figure and refer to it by that key. 
The \textbf{key is assigned a number} by the \verb:\label: command, and 
the \textbf{number is printed} by the \verb:\ref: command.

\subsection{Example}\label{example}

Euler's equation
\begin{equation}
   e^{i\pi} + 1 = 0 \label{eq:euler}
\end{equation}
combines the five most important numbers in mathematics in a single equation.

Equation \ref{eq:euler} in subsection \ref{example} is Euler's Famous result. \newline

The example above is created with: 
\begin{lstlisting}[language={[LaTeX]TeX}]

\subsection{Example}\label{example}

Euler's equation
\begin{equation}
   e^{i\pi} + 1 = 0 \label{eq:euler}
\end{equation}
combines the five most important numbers in mathematics in a single equation.

Equation \ref{eq:euler} in subsection \ref{example} is Euler's Famous result.

\end{lstlisting}

Where:
\begin{itemize}
   \item the \verb:\label{example}: binds the subsection number to the key
    {\tt example}
    
   \item the \verb-\label{eq:euler}- command binds the equation number to the key
    {\tt eq:euler} 
    
   \item \verb-\ref{eq:euler}- and \verb:\ref{example}: can be then used to reference the
    equation and the subsection
\end{itemize}


% ................................................
\subsection{Keys, Labels, Page References, Captions}

\subsubsection{Keys}

A key can consist of any sequence of letters, digits, or punctuation characters (Section 2.1). 
Upper- and lowercase letters are different, so ``gnu'' and ``Gnu'' would be distinct keys. 

In addition to sectioning commands, the following environments also generate numbers that can be
assigned to keys with a \verb:\label: command: 
{\tt equation}, 
{\tt eqnarray},
{\tt figure}, 
{\tt table}, 
{\tt enumerate} (this assigns the current item's number), 
and any theorem-like environment defined with the \verb:\newtheorem: command of Section 3.4.3.

Using keys for cross-referencing saves you from keeping track of the actual numbers, but it requires
you to remember the keys. 

\textbf{Possibly incorrect or outdated}: You can produce a list of the keys by running \LaTeX\ on the
 input file{\tt lablst.tex}. (You probably do this by typing ``latex lablst'' ; check your Local
 Guide to be sure.) \LaTeX\ will then ask you to type in the name of the input file whose keys you
 want listed, as well as the name of the document class specified by that file's 
 \verb:\documentclass: command.
 
 \textbf{One method that worked:} open the .aux file and look there\sepfootnote{produce-list-keys}.
 
% .......................
\subsubsection{Labels}

When a \verb:\label: command appears in ordinary text, it uses the number of the current
sectional unit as key. A \verb:\label: command appearing inside a numbered environment assigns
the environment's number to the key.

\begin{itemize}
   
   \item To assign the number of a sectional unit to a key put the \verb:\label: command in the
    argument of the sectioning command. (Technically, you can put the \verb:\label: command
    anywhere within the unit, except for a command argument or an environment, in which it would
    assign some other number.)

   \item To refer to an equation in an {\tt eqnarray} environment, put the \verb:\label: command 
    anywhere between 
    the \verb:\\: or \verb:\begin{eqnarray}: that begins the equation 
    and the \verb:\\: or \verb:\end{eqnarray}: that ends it

   \item The position of the \verb:\label: command in a figure or table \textbf{must} go after
    the \verb:\caption: command or in its argument
    
   \item A \verb:\label: can appear in the argument of a sectioning or \verb:\caption:
    command, but in no other moving argument

\end{itemize}

\paragraph{More detail:} The \verb:\label: command writes an entry on the aux file; this entry
contains the key, the current \verb:\ref: value, and the number of the current page. 
 
When this aux file entry is read by the \verb:\begin{document}: command\sepfootnote{firstrun}, 
the \verb:\ref: value and page number are associated with the \emph{key}. 
Then, a \verb:\ref{key}: or \verb:\pageref{key}: command will produce the associated 
\verb:\ref: value or page number, respectively.
 
The \verb:\label: command is fragile\sepfootnote{fragile}, but it can be used in the argument of
a sectioning or \verb:\caption: command.

% .......................
\subsubsection{Page References}

\LaTeX\ maintains a current \verb:\ref: value, which is set with the \verb:\refstepcounter: 
declaration (Section C.8.4). (This declaration is issued by the sectioning commands, by numbered
environments like {\tt equation}, and by an \verb:\item: command in an enumerate environment.)

The \verb:\pageref: command produces the page number where its corresponding \verb:\label: 
command appears. (A \verb:\ref: or \verb:\pageref: command generates only a number, so if you 
want to produce something like ``page 42'', you have to type the word `page'.)

The numbers generated by \verb:\ref: and \verb:\pageref: were assigned to their corresponding
keys the previous time you ran \LaTeX\ on your document. Thus, the printed output will be incorrect
if any of these numbers have changed. \LaTeX\ will warn you if this may have happened, in which
case you should run it again on the input file to make sure the cross-references are correct.
(This warning will occur if any number assigned to a key by a \verb:\label: command has changed,
even if that number is not referenced.) 

Each \verb:\ref: or \verb:\pageref: referring to an unknown key produces a warning
message; \emph{such messages appear the first time you process any file containing these commands}.

% .......................
\subsubsection{Captions} 

A \verb:\caption: command within its {\tt figure} or {\tt table} environment acts like a
sectioning command within the document. Just as a document has multiple sections, a figure or table
can have multiple captions.


% ..................................................................................................
\section{Bibliography and Citation}
% pdf 86, page 69

A \emph{citation} is a cross-reference to another publication, such as a journal article, called
the \emph{source}.

The modern method of citing a source is with a cross-reference
to an entry in a list of sources at the end of the document. 

With \LaTeX, the citation is produced by a \verb:\cite: command having the citation key as its
argument.

You can cite multiple sources with a single \verb:\cite: , separating the keys by commas. 
The \verb:\cite: command has an optional argument that adds a note to the citation.

\begin{lstlisting}[language={[LaTeX]TeX}]
This is a sentence~\cite{tom-ix, dick:ens, harry+d}
This is another sentence~\cite[pages 222--333]{kn:gnus}
\end{lstlisting}

A citation key can be any sequence of letters, digits, and punctuation characters, except that it
may not contain a comma. As usual in \LaTeX, upper- and lowercase letters are considered to be
different.

In the preceding examples, \LaTeX\ has to determine that citation key \verb-kn:gnus- corresponds to
source label 67. How \LaTeX\ does this depends on how you produce the list of sources. The best way
to produce the source list is with a separate program called BIB\TeX, described in Section 4.3.1.
You can also produce it yourself, as explained in Section 4.3.2.

% ................................................
\subsection{Using Bib\TeX}

Bib\TeX\ is a separate program that produces the source list for a document, obtaining the
information from a bibliographic database. 

\begin{itemize}
   
   \item To use BIB\TeX, you must include in your \LaTeX\ input file a \verb:\bibliography: command
    whose argument specifies one or more files that contain the database. The names of the database
    files must have the extension \texttt{bib}.
    
   \begin{itemize}
      \item For example, the command \verb:\bibliography{insect,animal}: (with \textbf{no space
       after the comma}) specifies that the source list is to be obtained from entries in the files
       insect.bib and animal.bib.
   \end{itemize}
   
   \item BIB\TeX\ creates a source list containing entries for all the citation keys specified
    by \verb:\cite: commands. 
   
   \begin{itemize}
      \item The data for the source list is obtained from the bibliographic database, which must
       have an entry for every citation key.
   \end{itemize}
    
   \item To use BIB\TeX, your \LaTeX\ input file must contain a \verb:\bibliographystyle: command.
    This command specifies the bibliography style, which determines the format of the source list. 
    The \verb:\bibliographystyle: command can go anywhere after the \verb:\begin{document}: command. 
    \LaTeX's standard bibliography styles are: plain, unsrt, alpha, abbrv.
   
   \item A \verb:\nocite: command in the \LaTeX\ input file causes the specified entries to appear
    in the source list, but produces no output. 
    It can go anywhere after the \verb:\begin{document}: command, but it is fragile.
    The command \verb:\nocite{*}: causes all entries in the bibliographic database to be put in the
    source list.
\end{itemize}

% Why would it go "anywhere". Shouldn't it go in under references?
% Appendix B explains how to make bibliographic database files.

% .......................
\subsubsection{Traditional Steps When Using Bib\TeX}
% I call them "traditional" because this is an old book, now there are other methods to do this
% but still it might be a good reference

Once you've created an input file containing the appropriate \LaTeX\ commands, you perform the
following sequence of steps to produce the final output:

\begin{itemize}
   \item Run \LaTeX\ on the input file (in this example called \texttt{myfile.tex}). \LaTeX\ will
    complain that all your citations axe undefined, since there is no source list yet.

   \item Run BIB\TeX\ by typing something like \texttt{bibtex myfile.tex}. BIB\TeX\ will generate
    the file \texttt{myfile.bbl} containing \LaTeX\ commands to produce the source list.
   
   \item Run \LaTeX\ again on \texttt{myfile.tex}. \LaTeX's output will now contain the source list.
    However, it will still complain that your citations are undefined, since the output produced by
    a \verb:\cite: command is based on information obtained from the source list the last time was
    run on the file.

   \item Run \LaTeX\ one more time on \texttt{myfile.tex}
\end{itemize}

If you add or remove a citation, you will have to go through this whole procedure again to get the
citation labels and source list right. But they don't have to be right while you're writing, so you
needn't do this very often

BIB\TeX\ almost always produces a perfectly fine source list, but if you encounter a source that it
does not handle properly you can usually correct the problem by modifying the bibliographic
database---perhaps creating a special database entry just for this document. As a last resort, you
can edit the \texttt{bbl} file that BIB\TeX\ generated. (Only do this when you are producing the
final output.)


% ..................................................................................................
\section{Splitting Your Input}
% pdf 89, page 72

A large document requires a lot of input. Rather than putting the whole input in a single large
file, you may find it more convenient to split the input into several smaller files. Regardless of
how many separate files you use, there is one that is the root file; it is the one whose name you
type when you run \LaTeX.

The \verb:\input: command provides the simplest way to split your input into several files. For
example, the command \verb:\input{gnu}: in the root file causes \LaTeX\ to insert the contents of a
secondary file {\tt gnu.tex} right at the current spot in your manuscript. The input files are not
changed, and the secondary file {\tt gnu.tex} may also contain an \verb:\input: command calling
another file that may have its own \verb:\input: commands, and so on.

Besides splitting your input into conveniently sized chunks, the \verb:\input: command also makes it
easy to use the same input in different documents. While text is seldom recycled in this way, you
might want to reuse declarations. You can keep a file containing declarations that are used in all
your documents, such as the definitions of commands and environments for your own logical
structures (Section 3.4). You can even begin your root file with an \verb:\input: command and put
the \verb:\documentclass: command in your declarations file.

[Continue reading on page 73 to learn about how to use the \verb:\include: command, instead
of \verb:\input:, to only run \LaTeX\ on part of the document. (But this might be more trouble than
it's worth.)]

% ..................................................................................................
\section{Making an Index or Glossary}

There are two steps in making an index or glossary: gathering the information that goes in it, and
generating the \LaTeX\ input to produce it. 
Section \ref{compiling-entries} describes the first step. 
Section \ref{manual-production-index} describes the second step if you don't use the 
\emph{Makeindex} program (which is the easiest way to do this, and is separately described in 
Appendix A.)

% ................................................
\subsection{Compiling the entries}\label{compiling-entries}

Compiling an index or glossary is not easy but \LaTeX\ can help by writing the necessary information
onto a special file. If the root file is named {\tt myfile.tex}, index information will be written
on a file called {\tt myfile.idx}. 

\LaTeX\ makes an {\tt idx} file if the preamble contains a \verb:\makeindex: command; and the 
\verb:\index: commands write information to the file. 
(If there is no \verb:\makeindex: command, the \verb:\index: command does nothing.)

% pdf 92, page 74

% All the code below is to make a title "Example" followed by a normal looking paragraph 
% Technically "Example" is one paragraph (it is indented) and "To index ..." is another; so indents 
% must be removed (noindent), and a space added between the paragraphs (bigskip)

\subsubsection*{Example}

To index an instance of the word ``gnu'' in the phrase ``The gnu drinks water from the river'', you 
would type \verb:The gnu\index{gnu} drinks water from the river:.\\

If this word were in page 42, the command \verb:\index{gnu}: causes \LaTeX\ to
write \verb:\indexentry{gnu}{42}: on the {\tt idx} file. (It's best to put the \verb:\index:
command next to the word it references, with no space between them; this keeps the page number from
being off by one if the word ends or begins a page.)

\subsubsection*{Example}

In this example, the \% ends a line without producing any space in the output (but you can't split a
command name across two lines).
 
\begin{verbatim}
When in the Course of
\index{human events)%
\index{events, human)%
human events, it becomes necessary for one people to dissolve the political bands ...
\end{verbatim}

The procedure for making a glossary is completely analogous: there is a \verb:\makeglossary: command
to make a {\tt glo} file which has \verb:\glossaryentry: entries created by a \verb:\glossary:
command.

The \verb:\index: and \verb:\glossary: commands are fragile\sepfootnote{fragile}. Moreover,
an \verb:\index: or \verb:\glossary: command should not appear in the argument of any other command
if its own arguments contains any of \LaTeX\ ten special characters.

% ................................................
\subsection{Producing an Index or Glossary by Yourself}\label{manual-production-index}

If you don't use the \emph{Makeindex} program, you can use the {\tt theindex} environment to produce
an index in two-column format. In this scenario:

\begin{itemize}
   \item Each main index entry is begun by an \verb:\item: command
   \item A subentry is begun with \verb:\subitem:, and a subsubentry is begun with \verb:\subsubitem:
   \item Blank lines between entries are ignored
   \item An extra vertical space is produced by the \verb:\indexspace: command, which is usually
    put before the first entry that starts with a new letter
\end{itemize}

\subsubsection*{Example}

\begin{verbatim}
\item gnats 13, 97
\item gnus 24, 37, 233
\subitem bad, 39, 236
\subsubitem very, 235
\subitem good, 38, 234
\indexspace
\item harmadillo 99, 144
\end{verbatim}

There is no environment expressly for glossaries. (However, the description environment of Section
2.2.4 may be useful.)


% --------------------------------------------------------------------------------------------------
% --------------------------------------------------------------------------------------------------
\end{document}               % End of document.
% --------------------------------------------------------------------------------------------------
% --------------------------------------------------------------------------------------------------
