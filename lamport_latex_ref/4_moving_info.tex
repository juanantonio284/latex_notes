\documentclass{article}

% --------------------------------------------------------------------------------------------------
% PREAMBLE START

% my commands .............................................................
\newcommand{\justtext}[1]{\texttt{\textbackslash #1}}

% separate footnotes ......................................................
\usepackage{sepfootnotes} % https://ctan.org/pkg/sepfootnotes

\sepfootnotecontent{error}{
An error in your input file could produce an error in one of the special cross-referencing files.
The error in the cross-referencing file will not manifest itself until that file is read, the next
time you run \LaTeX. Section 8.1 explains how to recognize such an error.
}

\sepfootnotecontent{firstrun}{the next time \LaTeX\ is run on the same input file}

% title ...................................................................
\title{Chapter 4: Moving Information Around} % Declares the document's title.
\author{}                                 
\date{}                               
% Note that if you simply omit the `\date` command, the current date will still be printed. To avoid printing a date insert the command `\date{ }`, i.e. the command with blank arguments.

% --------------------------------------------------------------------------------------------------
% PREAMBLE END


\begin{document}
% --------------------------------------------------------------------------------------------------
% --------------------------------------------------------------------------------------------------

\maketitle % Produces the title.

\LaTeX\ often has to move information from one place to another. For example, the information
contained in a table of contents comes from the sectioning commands that are scattered throughout
the input file. Similarly, the command that generates a cross-reference to an equation must get the
equation number from the {\tt equation} environment, which may occur several sections later. 

\LaTeX\ requires two passes over the input to move information around: one pass to find the
information and a second pass to put it into the text—it occasionally even requires a third pass. To
compile a table of contents, for example, one pass determines the titles and starting pages of all
the sections, and a second pass puts this information into the table of contents. But instead of
making two passes every time it is run, \LaTeX\ reads your input file only once and saves the
cross-referencing information in special files for use the next time. 

For example: if {\tt sample2e.tex} had a command to produce a table of contents \LaTeX\ would write
the necessary information into the file {\tt sample2e.toc}; then it would use the information in
that file to produce the table of contents in the typeset document. After that, the
{\tt sample2e.toc} file would be re-written to a newer version (this version will be used to
produce the table of contents the next time \LaTeX\ is run) on the tex file. But notice that this
``newer'' version will actually be a version from a previous execution, meaning that \LaTeX's
cross-referencing information is always ``old''. This will be noticeable mainly when you are first
writing the document—for example, a newly added section won't be listed in the table of
contents—but running it again on the same input will correct any errors\sepfootnote{error}.


% ..................................................................................................
\section{Cross-References}

One reason for numbering things like figures and equations is to refer the reader to them, as in:
``See Figure 3 for more details''. You don't want the character `3' to actually be written in the
input file because adding another figure might make the figure become ``Figure 4'', and require
changing the `3' to a `4' everywhere it appears. Instead, you can assign a \emph{key} of your
choice to the figure and refer to it by that key. 
The \textbf{key is assigned a number} by the \justtext{label} command, and 
the \textbf{number is printed} by the \justtext{ref} command.

In the example below:

\begin{itemize}
   \item the \justtext{label\{example\}} binds the subsection number to the key
    {\tt example}
    
   \item the \justtext{label\{eq:euler\}} command binds the equation number to the key
    {\tt eq:euler} 
    
   \item \justtext{ref\{eq:euler\}} and \justtext{ref\{example\}} can be then used to reference the
    equation and the subsection
\end{itemize}

\subsection{Example}\label{example}

Euler's equation
\begin{equation}
   e^{i\pi} + 1 = 0 \label{eq:euler}
\end{equation}
combines the five most important numbers in mathematics in a single equation.

Equation \ref{eq:euler} in subsection \ref{example} is Euler's Famous result. 

% ................................................
\subsection{Keys, Labels, Page References, Captions}

\subsubsection{Keys}

A key can consist of any sequence of letters, digits, or punctuation characters (Section 2.1). 
Upper- and lowercase letters are different, so ``gnu'' and ``Gnu'' would be distinct keys. In
addition to sectioning commands, the following environments also generate numbers that can be
assigned to keys with a \justtext{label} command: 
{\tt equation}, 
{\tt eqnarray},
{\tt figure}, 
{\tt table}, 
{\tt enumerate} (this assigns the current item's number), 
and any theorem-like environment defined with the \justtext{newtheorem} command of Section 3.4.3.

Using keys for cross-referencing saves you from keeping track of the actual numbers, but it requires
you to remember the keys. 

\textbf{Possibly incorrect or outdated}: You can produce a list of the keys by running \LaTeX\ on the
 input file{\tt lablst.tex}. (You probably do this by typing ``latex lablst'' ; check your Local
 Guide to be sure.) \LaTeX\ will then ask you to type in the name of the input file whose keys you
 want listed, as well as the name of the document class specified by that file's 
 \justtext{documentclass} command.
 
 \textbf{One method that worked:} open the .aux file and look 
 there\footnote{https://tex.stackexchange.com/questions/525394/get-a-list-of-all-labels-in-a-tex-document}.
 
% .......................
\subsubsection{Labels}

When a \justtext{label} command appears in ordinary text, it uses the number of the current
sectional unit as key. A \justtext{label} command appearing inside a numbered environment assigns
the environment's number to the key.

\begin{itemize}
   
   \item To assign the number of a sectional unit to a key put the \justtext{label} command in the
    argument of the sectioning command. (Technically, you can put the \justtext{label} command
    anywhere within the unit, except within a command argument or environment in which it would
    assign some other number.)

   \item To refer to an equation in an {\tt eqnarray} environment, put the \justtext{label} command 
    anywhere between the \justtext{\textbackslash} or \justtext{begin\{eqnarray\}} that begins the 
    equation and the \justtext{\textbackslash} or \justtext{end\{eqnarray\}} that ends it

   \item The position of the \justtext{label} command in a figure or table \textbf{must} go after
    the \justtext{caption} command or in its argument
    
   \item A \justtext{label} can appear in the argument of a sectioning or \justtext{caption} 
    command, but in no other moving argument

\end{itemize}

\paragraph{More detail:} The \justtext{label} command writes an entry on the aux file; this entry
contains the key, the current \justtext{ref} value, and the number of the current page. 
 
When this aux file entry is read by the \justtext{begin\{document\}} command\sepfootnote{firstrun}, 
the \justtext{ref} value and page number are associated with the \emph{key}. 
Then, a \justtext{ref\{key\}} or \justtext{pageref\{key\}} command will produce the associated 
\justtext{ref} value or page number, respectively.
 
The \justtext{label} command is fragile, but it can be used in the argument of a sectioning
or \justtext{caption} command.

% .......................
\subsubsection{Page References}

\LaTeX\ maintains a current \justtext{ref} value, which is set with the \justtext{refstepcounter} 
declaration (Section C.8.4). (This declaration is issued by the sectioning commands, by numbered
environments like {\tt equation}, and by an \justtext{item} command in an enumerate environment.)

The \justtext{pageref} command produces the page number where its corresponding \justtext{label} 
command appears. (A \justtext{ref} or \justtext{pageref} command generates only a number, so if you 
want to produce something like ``page 42'', you have to type the word `page'.)

The numbers generated by \justtext{ref} and \justtext{pageref} were assigned to their corresponding
keys the previous time you ran \LaTeX\ on your document. Thus, the printed output will be incorrect
if any of these numbers have changed. \LaTeX\ will warn you if this may have happened, in which
case you should run it again on the input file to make sure the cross-references are correct.
(This warning will occur if any number assigned to a key by a \justtext{label} command has changed,
even if that number is not referenced.) 

Each \justtext{ref} or \justtext{pageref} referring to an unknown key produces a warning
message; \textbf{such messages appear the first time you process any file containing these
commands}.

% .......................
\subsubsection{Captions} 

A \justtext{caption} command within its {\tt figure} or {\tt table} environment acts like a
sectioning command within the document. Just as a document has multiple sections, a figure or table
can have multiple captions.



% ..................................................................................................
% \section{}



% --------------------------------------------------------------------------------------------------
% --------------------------------------------------------------------------------------------------
\end{document}               % End of document.
% --------------------------------------------------------------------------------------------------
% --------------------------------------------------------------------------------------------------
