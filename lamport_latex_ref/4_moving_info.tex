\documentclass{article}

% --------------------------------------------------------------------------------------------------
% PREAMBLE START

% my commands .............................................................
\newcommand{\justtext}[1]{\texttt{\textbackslash #1}}

% separate footnotes ......................................................
\usepackage{sepfootnotes} % https://ctan.org/pkg/sepfootnotes

\sepfootnotecontent{error}{
An error in your input file could produce an error in one of the special cross-referencing files.
The error in the cross-referencing file will not manifest itself until that file is read, the next
time you run \LaTeX. Section 8.1 explains how to recognize such an error.
}

% title ...................................................................
\title{Chapter 4: Moving Information Around} % Declares the document's title.
\author{}                                 
\date{}                               
% Note that if you simply omit the `\date` command, the current date will still be printed. To avoid printing a date insert the command `\date{ }`, i.e. the command with blank arguments.

% --------------------------------------------------------------------------------------------------
% PREAMBLE END


\begin{document}
% --------------------------------------------------------------------------------------------------
% --------------------------------------------------------------------------------------------------

\maketitle % Produces the title.

\LaTeX\ often has to move information from one place to another. For example, the information
contained in a table of contents comes from the sectioning commands that are scattered throughout
the input file. Similarly, the command that generates a cross-reference to an equation must get the
equation number from the {\tt equation} environment, which may occur several sections later. 

\LaTeX\ requires two passes over the input to move information around: one pass to find the
information and a second pass to put it into the text—it occasionally even requires a third pass. To
compile a table of contents, for example, one pass determines the titles and starting pages of all
the sections and a second pass puts this information into the table of contents. But instead of
making two passes every time it is run, \LaTeX\ reads your input file only once and saves the
cross-referencing information in special files for use the next time. 

For example, if{\tt sample2e.tex} had a command to produce a table of contents, then \LaTeX\ would
write the necessary information into the file {\tt sample2e.toc}; then it would use the information
in that file to produce the table of contents in the typeset document. After that, the
{\tt sample2e.toc} is re-written to a newer version which will be used to produce the table of
contents the next time \LaTeX\ is run with{\tt sample2e.tex} as input. But notice that this
``newer'' version will actually be a version from a previous execution, \LaTeX's cross-referencing
information is therefore always old. This will be noticeable mainly when you are first writing the
document—for example, a newly added section won't be listed in the table of contents—but running it
again on the same input will correct any errors\sepfootnote{error}.

% ..................................................................................................
\section{Cross-References}

One reason for numbering things like figures and equations is to refer the reader to them, as
in: ``See Figure 3 for more details''. You don't want the ``3'' to appear in the input file because
adding another figure might make this one become Figure 4. Instead, you can assign a \emph{key} of
your choice to the figure and refer to it by that key, letting \LaTeX\ translate the reference into
the figure number. The key is assigned a number by the \justtext{label} command, and the number is
printed by the \justtext{ref} command.

A \justtext{label} command appearing in ordinary text assigns to the key the number of the current
sectional unit; one appearing inside a numbered environment assigns that number to the key. 
\textbf{Example}: equation \ref{eq:euler} in subsection \ref{sample_subsection} below is Euler's
Famous result. 

\subsection{Example}\label{sample_subsection}

Euler's equation
\begin{equation}
   e^{i\pi} + 1 = 0 \label{eq:euler}
\end{equation}
combines the five most important numbers in mathematics in a single equation.

In this example, the \justtext{label\{eq:euler\}} command assigns the key {\tt eq:euler} to the
equation number, and \justtext{ref\{eq:euler\}} generates that equation number.

% ................................................
\subsection{Keys, Labels, Captions, pageref, ref}

\paragraph{Keys:} A key can consist of any sequence of letters, digits, or punctuation characters
 (Section 2.1). Upper- and lowercase letters are different, so gnu and Gnu are distinct keys. In
 addition to sectioning commands, the following environments also generate numbers that can be
 assigned to keys with a \justtext{label} command: 
 {\tt equation}, 
 {\tt eqnarray},
 {\tt enumerate} (assigns the current item's number), 
 {\tt figure}, 
 {\tt table}, 
 and any theorem-like environment defined with the \justtext{newtheorem} command of Section 3.4.3.

\paragraph{Labels:} The \justtext{label} command can usually go in any natural place. To assign the
 number of a sectional unit to a key, you can put the \justtext{label} command anywhere within the
 unit except within a command argument or environment in which it would assign some other number,
 or you can put it in the argument of the sectioning command. 

To refer to a particular equation in an eqnarray environment, put the \justtext{label} command
anywhere between the \justtext{\textbackslash} or \justtext{begin\{eqnarray\}} that begins the
equation and the \justtext{\textbackslash} or \justtext{end\{eqnarray\}} that ends it.

The position of the \label command in a figure or table is less obvious: it must go after
the \justtext{caption} command or in its argument.

% .......................
\paragraph{Captions:} A \justtext{caption} command within its {\tt figure} or {\tt table}
 environment acts like a sectioning command within the document. Just as a document has multiple
 sections, a figure or table can have multiple captions.

% .......................
\paragraph{Page reference:} The \justtext{pageref} command is similar to the \justtext{ref} command
 except it produces the page number of the place in the text where the corresponding \justtext
 {label} command appears. 

A \justtext{ref} or \justtext{pageref} command generates only the number, so you have to type the
page to produce something like ``page 42''.

The numbers generated by \justtext{ref} and \justtext{pageref} were assigned to the keys the
previous time you ran \LaTeX\ on your document. Thus, the printed output will be incorrect if any
of these numbers have changed. \LaTeX\ will warn you if this may have happened, in which case you
should run it again on the input file to make sure the cross-references are correct. (This warning
will occur if any number assigned to a key by a \justtext{label} command has changed, even if that
number is not referenced.) Each \justtext{ref} or \justtext{pageref} referring to an unknown key
produces a warning message; such messages appear the first time you process any file containing
these commands.

A \justtext{label} can appear in the argument of a sectioning or \justtext{caption} command, but in
no other moving argument.

Using keys for cross-referencing saves you from keeping track of the actual numbers, but it requires
you to remember the keys. You can produce a list of the keys by running \LaTeX\ on the input file
lablst.tex. (You probably do this by typing ``latex lablst'' ; check your Local Guide to be
sure.) \LaTeX\ will then ask you to type in the name of the input file whose keys you want listed,
as well as the name of the document class specified by that file's \justtext{documentclass} command.

% ..................................................................................................
% \section{}



% --------------------------------------------------------------------------------------------------
% --------------------------------------------------------------------------------------------------
\end{document}               % End of document.
% --------------------------------------------------------------------------------------------------
% --------------------------------------------------------------------------------------------------
