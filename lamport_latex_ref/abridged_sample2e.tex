% The original "sample2e.tex" file comes with the standard? latex installation
% you should have it in your computer at /usr/share/texlive/texmf-dist/tex/latex/base/sample2e.tex
% This is an abridged version with only the notes/lessons I found interesting

\documentclass{article}      % Specifies the document class

                                   % The preamble begins here.
\title{Lessons taken from the {\tt sample2e.tex} file}  % Declares the document's title.
\author{juanantonio284}                                 % Declares the author's name.
\date{February 7, 2025}                               
                                                        % Note that if you simply omit the `\date`
                                                        % command, the current date will still be
                                                        % printed. To avoid printing a date insert
                                                        % the command `\date{ }`, i.e. the command
                                                        % with blank arguments.
  
\newcommand{\ip}[2]{(#1, #2)}
                             % Defines \ip{arg1}{arg2} to mean
                             % (arg1, arg2).

%\newcommand{\ip}[2]{\langle #1 | #2\rangle}
                             % This is an alternative definition of
                             % \ip that is commented out.


% --------------------------------------------------------------------------------------------------
% --------------------------------------------------------------------------------------------------
\begin{document}             % End of preamble and beginning of text.
% --------------------------------------------------------------------------------------------------
% --------------------------------------------------------------------------------------------------

\maketitle                   % Produces the title.

This is an example input file. Comparing it with the output it generates can show you how to produce
a simple document of your own.

% --------------------------------------------------------------------------------------------------
\section{Ordinary Text}      % Produces section heading.  Lower-level
                             % sections are begun with similar
                             % \subsection and \subsubsection commands.

Since any number of consecutive spaces are treated like a single one, the formatting of the input file makes no difference to \LaTeX, but it makes a difference to you ...

Because printing is different from typewriting, there are a number of things that you have to do
differently when preparing an input file than if you were just typing the document directly.

\paragraph{Quotation Marks} Quotation marks like ``this'' have to be handled specially, as do quotes
 within quotes: ``\,`this' is what I just wrote, not `that'\,''. % \, separates the double and
 single quote.

\paragraph{Dashes} Dashes come in three sizes: an intra-word dash, a medium dash for number ranges
 like 1--2, and a punctuation dash---like this. The ``punctuation dash'' is the em-rule
 (unicode 2014)—to make it, you can directly enter the unicode symbol, or use three regular hyphens
 one after another {\tt ---}.

\paragraph{Sentence ending space vs.\ intra-word space}

In traditional typesetting, sentence-ending space should be larger than the space between words
within a sentence. \TeX\ also follows this custom. But how does \TeX\ know whether a period ends a
sentence or not? It assumes that every period not following an upper case letter ends a
sentence\footnote{The logic is: if it follows an upper case letter then it's likely an abbreviation
and not the end of a sentence.}.

\begin{itemize}

\item In instances where a period following a lowercase letter does not end a sentence 
      \textbf{enter an escaped space ({\tt \textbackslash \textvisiblespace }) to avoid a larger
       space}. [The compiler focuses on the space not on the period, and thus does not add the
       larger space]. 
       % {\tt } is a "typewriter font" environment
       % \textbackslash is a command to print the \ symbol
       % \textvisiblespace is a command to print a symbol to indicate a blank space

\item In instances where a sentence ends in an upper case letter 
      \textbf{add an extra space after the period with {\tt \textbackslash @}}.

\item \emph{Example}: The words `gnat', `gnus', `gnome', etc.\ all begin with G\@.
       % `\ ' makes an inter-word space. \@ marks end-of-sentence punctuation.

\end{itemize}

You should check the spaces after periods when reading your output to make sure you haven't
forgotten any special cases.  

\paragraph{Ellipsis} Generating an ellipsis \ldots\ with the right spacing around the periods
 requires a special command.
% `\ ' is needed after `\ldots' because TeX ignores spaces after command names like \ldots made
% from \ + letters. 

\paragraph{Characters that need escaping} \LaTeX\ interprets some common characters as commands, so
 you must type special commands to generate them.  These characters include the following:
       \$ \& \% \# \{ and \}.

\paragraph{Emphasizing text} In printing, text is usually emphasized with an \emph{italic} type
 style.

\begin{em}
   A long segment of text can also be emphasized
   in this way.  Text within such a segment can be
   given \emph{additional} emphasis.
\end{em}

\paragraph{Prevent line breaks} It is sometimes necessary to prevent \LaTeX\ from breaking a line
 where it might otherwise do so. This may be at a space, as between the ``Mr.''\ and ``Jones'' in
 ``Mr.~Jones'', or within a word---especially when the word is a symbol like 
 \mbox{\emph{itemnum}} that makes little sense when hyphenated across lines. 
 % ~ produces an unbreakable interword space.

\paragraph{Footnotes} Footnotes\footnote{This is an example of a footnote.} pose no problem.

\paragraph{Formulas} \LaTeX\ is good at typesetting mathematical formulas like 
       \( x-3y + z = 7 \)
or 
       \( a_{1} > x^{2n} + y^{2n} > x' \) 
or 
       \( \ip{A}{B} = \sum_{i} a_{i} b_{i} \). 
The spaces you type in a formula are ignored. Remember that a letter like $x$ is a formula when it
denotes a mathematical symbol, and it should be typed as one.
 % $ ... $  and  \( ... \)  are equivalent


% --------------------------------------------------------------------------------------------------
\section{Displayed Text}

\subsection{Quotations with the \emph{quote} and \emph{quotation} environments}

Text is displayed by indenting it from the left margin.  Quotations are commonly displayed.  There
are short quotations
\begin{quote}
   This is a short quotation.  It consists of a single paragraph of text.  See how it is formatted.
\end{quote}
and longer ones.
\begin{quotation}
   This is a longer quotation.  It consists of two paragraphs of text, neither of which are
   particularly interesting.

   This is the second paragraph of the quotation.  It is just as dull as the first paragraph.
\end{quotation}

% ------------------------------------------------
\subsection{Lists with the \emph{itemize} and \emph{enumerate} environments}

Another frequently-displayed structure is a list. The following is an example of an 
\emph{itemized} list.

\begin{itemize}
   
   \item This is the first item of an itemized list. Each item in the list is marked with a
    ``tick''—usually a bullet (•), but you don't have to worry about what kind of tick mark is
    used.

   \item This is the second item of the list.  It contains another list nested inside it.  The inner
    list is an \emph{enumerated} list.
         \begin{enumerate}
            \item This is the first item of an enumerated list that is nested within the itemized
             list.

            \item This is the second item of the inner list. \LaTeX\ allows you to nest lists deeper
             than you really should.
         \end{enumerate}
         This is the rest of the second item of the outer list.  It is no more interesting than any
         other part of the item.
   
   \item This is the third item of the list.
\end{itemize}

% ------------------------------------------------
\subsection{Poetry with the \emph{verse} environment}

You can even display poetry.

\begin{verse}
   There is an environment for verse \\             % The \\ command separates lines within a stanza
   Whose features some poets will curse.

                             % One or more blank lines separate stanzas.

   For instead of making\\
   Them do \emph{all} line breaking, \\
   It allows them to put too many words on a line when they'd rather be forced to be terse.
\end{verse}

% ------------------------------------------------
\subsection{Mathematical formulas}

Mathematical formulas may also be displayed.  A displayed formula is one-line long; multiline
formulas require special formatting instructions.
   \[  \ip{\Gamma}{\psi'} = x'' + y^{2} + z_{i}^{n}\]

Don't start a paragraph with a displayed equation, nor make one a paragraph by itself.

% --------------------------------------------------------------------------------------------------
% --------------------------------------------------------------------------------------------------
\end{document}               % End of document.
% --------------------------------------------------------------------------------------------------
% --------------------------------------------------------------------------------------------------
